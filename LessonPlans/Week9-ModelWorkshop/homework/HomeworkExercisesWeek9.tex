\documentclass[11pt]{homework}


\title{Complex Sample Designs}
\assignment{Homework Exercises, Week 9}
\duedate{April 1, 2018}
\coursetitle{Practicum in Sociological Research}
\coursenumber{SOCY 625}
\author{Prof. Michael Bader}

\usepackage{enumitem}

%\printanswers
\begin{document}
\maketitle 

\begin{questions}
\uplevel{\bfseries The following questions refer to chapter 4 in Groves, et al.~(2009).}

\question[2] An investigator has started thinking about conducting a study so that they can identify exposure to gun violence from a nationally representative sample of respondents. They call you in to ask you what the pros and cons are of using a clustered sample. Write below the list of pros and cons you would tell the investigator. 

\question[2] The Vice President of Campus Life has asked you to conduct a survey of students living on campus about residential life. List for her the pros and cons of using a sample stratified by dorm to represent students who live on campus. 

\question[5] Answer Question 3 on page 139 from Groves, et al.\ (parts a-e). 
\begin{solution}
\begin{enumerate}[label=\alph*]
\item The design effect will be larger using a clustered sample, especially if the within-cluster homogeneity is high
\item The design effect of a stratified sample will generally be lower than one not using stratification
\item The intraclass correlation will be close to 1. 
\item Need to calculate
\item Need to explain 
\end{enumerate}
\end{solution}

\question Assume that the DCAS 2018 will have an effective sample sample-size (epsem equivalent) to 1,000 respondents. Estimate the margin of error on your dependent variable that you would estimate based on the steps below. 
\begin{parts}
\part[1] Estimate what the mean and variance of your dependent variable will be
\part[1] Calculate the standard error
\part[3] Calculate the 95\% confidence interval around your estimated mean for an epsem sample size of 1,000 (you should use either $\pm$2 s.e. or $\pm$1.97 s.e. for your estimates) 
\end{parts}

\uplevel{\bfseries Use the DCAS 2016 dataset to answer the following questions.}

\question[10] Using the responses to the original questions (i.e., those beginning with `q'), create variables representing the following (\emph{hint:} You can check your answers against the variable \texttt{dem.race.w.other}): 
\begin{parts}
\part Non-Hispanic white alone
\part Non-Hispanic black alone or in combination with other races
\part Non-Hispanic Asian or Pacific Islander alone or in combination with white
\part Hispanic
\part Other
\end{parts}


\question \label{q:props} Use the responses to Question 41 to answer the following: 
\begin{parts}
\part[1] Calculate the proportion of \emph{all respondents} who say that they are at least a little afraid that they will be targeted by the police for questioning or arrest. 
\part[5] Now calculate the proportion of respondents from each racial group who say that they are at least a little afraid that they will be targeted by the police for questioning or arrest. 
\end{parts}

\question[6] Pick a variable that interests you from the DCAS 2016 and repeat the exercises from Question~\ref{q:props} above.


\end{questions}
\end{document}
