\documentclass[11pt]{homework}


\title{Question Design}
\assignment{Homework Exercises, Week 5}
\duedate{February 26, 2018}
\coursetitle{Practicum in Sociological Research}
\coursenumber{SOCY 625}
\author{Prof. Michael Bader}



%\printanswers
\begin{document}
\maketitle 

\begin{questions}
\uplevel{\bfseries Answer the following questions on page 212 of Groves, et al. (2009).}

\question[4] Question 1 (note that the directions should state, ``Compute the response rate in \emph{four} different ways$\ldots$''). 

\question[3] Question 2.

\uplevel{\bfseries Answer the following questions on page 225 of Groves, et al.\ (2009).}

\question[6] 
Pick an \emph{attitude} related to your construct. Using what you know about constructing attitude questions, write a standardized, self-administered question that you think will capture the direction and strength of the attitude you wish to study. The follow the instructions in Question 2 on page 255 substituting the attitude you wish to study in place of attitudes about invading Iraq. 
\label{q:attitude}

\question[4] Follow the instructions in Question 3 on page 225 about the question that you constructed in Question~\ref{q:attitude}. 

\question[2] Question 6 (all four examples)

\question[2] Question 8 (a \& b only)

\uplevel{\bfseries Answer the following when thinking about a survey measuring the prevalence of predatory financial behavior targeted at elders with a target population of adults 65+ living in D.C. that Rachel Breslin discussed last week.}
\question Thinking about the target population and the mode of interview, what concerns would you have about:
\begin{parts}
\part[2] unit non-response 
\part[2] item non-response 
\end{parts}
\question[6] Using what you know about constructing \emph{behavior} questions, write a standardized, interviewer-administered question that asks elderly respondents to report whether they have ever given bank account or credit card information over the phone. 
\label{q:behavior} 
\question[2] Based on figure 7.1 on page 218 of Groves, et al.\ (2009), explain what processes a respondent would be required to perform to answer the question you constructed above. Be sure to refer to all four parts. 
\end{questions}
\end{document}
