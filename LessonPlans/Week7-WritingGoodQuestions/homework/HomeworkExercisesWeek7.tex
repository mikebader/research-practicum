\documentclass[11pt]{homework}


\title{Sampling Variance}
\assignment{Homework Exercises, Week 7}
\duedate{March 18, 2018}
\coursetitle{Practicum in Sociological Research}
\coursenumber{SOCY 625}
\author{Prof. Michael Bader}



%\printanswers
\begin{document}
\maketitle 

\begin{questions}
\uplevel{Please answer all parts of the the following two questions. There are no problems from the book this week}

\question[14] Please examine your own and your classmates' questions on Piazza. Use what you have learned in lecture, from Chapters 2, 7, and 8 of Groves, et al., and from my comments to you on your previous questions to evaluate how well the questions are designed to be fielded as part of the DCAS~2018. I would like you focus on the \textbf{question wording} using the checklist handout that I gave out based on recommendations in Groves, et al., and devote special attention to the: 
\begin{description}
\item[validity] of the question to measure the underlying construct, 
\item[bias] that might occur 
\item[non-response] that the question might induce at either the item level (respondents skipping the question) or at the unit level (respondents refusing to mail the survey back)
\end{description}

Please make what you write as helpful to your classmates as possible (imagine you were reading what you wrote). If you classmate included a block of questions, you may evaluate the block of questions together. 

\question Write out a \textbf{model} that you intend to analyze using the DCAS 2018. 
\begin{parts}
\part[2] Name your dependent variable(s) and make your best guess (hypothesis) at what the \emph{population mean} of that variable will be

\emph{NOTE:} If you are using a scale, decide whether you want to use the scale as a continuous variable or divide the scale into two parts (e.g., combine ``strongly agree'' and ``somewhat agree'' and estimate the proportion who somewhat or strongly agree--see the 2016 DCAS report for examples). Alternatively, you could number values from 1 to 5 and estimate the mean value from those answers 

\part[2] Name your main independent (predictor) variable and make your best guess (hypothesis) at what the \emph{population mean} of that variable will be (you may want to look at the demographic variables that we included in the DCAS 2016 for ideas)

\part[1] Write out the relationship of your independent variable to your dependent variable using $y\longleftarrow x$ notation. 

\part[2] Write out control variables that you will use to adjust the model to identify the influence of your independent variable on your dependent variable
\end{parts}


\end{questions}
\end{document}
