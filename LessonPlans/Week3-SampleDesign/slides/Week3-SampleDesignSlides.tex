\documentclass[]{beamer}
%\documentclass{article}
%\usepackage{beamerarticle}

\mode<presentation>
{
  \usetheme{Warsaw}
  % or ...

  \setbeamercovered{transparent}
  % or whatever (possibly just delete it)
}


\usepackage[english]{babel}
% or whatever

\usepackage[utf8]{inputenc}
% or whatever

\usepackage{times}
\usepackage[T1]{fontenc}
\usepackage{tikz}
\usetikzlibrary{arrows.meta}

%%%%%%%%%%%%%%%%%%%%%%%%%%%%%%%%%%
%%                         PACKAGES                      %%
%%%%%%%%%%%%%%%%%%%%%%%%%%%%%%%%%%
\usepackage{hyperref,ctable}
\usepackage{graphicx}
\usepackage{tikz}                    % For flowchart
\usetikzlibrary{shapes,arrows} % For flowchart


%%%%%%%%%%%%%%%%%%%%%%%%%%%%%%%%%%
%%                           COLORS                        %%
%%%%%%%%%%%%%%%%%%%%%%%%%%%%%%%%%%
%AU COLORS
\definecolor{aublue}{RGB}{25,33,129}
\definecolor{aured}{RGB}{244,28,31}

%Red
%\setbeamercolor{titlelike}{bg=aured,fg=white}
\setbeamercolor{structure}{bg=black!25!aured, fg=aured}
%\setbeamercolor*{palette primary}{fg=white,bg=aured}
%\setbeamercolor*{palette quaternary}{fg=white,bg=aublue}

%Blue
\setbeamercolor{titlelike}{bg=aublue,fg=white}
%\setbeamercolor{structure}{bg=black!25!aublue, fg=aublue}
\setbeamercolor*{palette primary}{fg=white,bg=aublue}
\setbeamercolor*{palette quaternary}{fg=white,bg=black!75!aublue}

\setbeamercolor{local structure}{fg=aured,bg=gray!60!aured}
\setbeamercolor{alerted text}{fg=aured}

\newenvironment{concept}[1]%
	{
	\setbeamercolor{background canvas}{bg=aured!10!white}%
	\setbeamercolor{frametitle}{bg=aured}%
	\setbeamercolor{frametitle right}{bg=aured}
	\setbeamercolor{alerted text}{fg=aured}%
	\begin{frame}{Concept}%
	\alert{\bfseries \large #1\\[2em]}}{%
	\end{frame}%
	}


%%%%%%%%%%%%%%%%%%%%%%%%%%%%%%%%%%
%%                         GRAPHICS                       %%
%%%%%%%%%%%%%%%%%%%%%%%%%%%%%%%%%%
%\graphicspath{{/Users/bader/work/Presentations/Images/}}}}


%%%%%%%%%%%%%%%%%%%%%%%%%%%%%%%%%%
%%                        COMMANDS                       %%
%%%%%%%%%%%%%%%%%%%%%%%%%%%%%%%%%%
\newcommand{\strong}[1]{\textbf{#1}}
\AtBeginSection[]{
  \begin{frame}
  \vfill
  \centering
  \begin{beamercolorbox}[sep=8pt,center,shadow=true,rounded=true]{title}
    \usebeamerfont{title}\insertsectionhead\par%
  \end{beamercolorbox}
  \vfill
  \end{frame}
}



%%%%%%%%%%%%%%%%%%%%%%%%%%%%%%%%%%
%%                     PRESENTATION                   %%
%%%%%%%%%%%%%%%%%%%%%%%%%%%%%%%%%%
\title{Coverage}

\author[Bader--SOCY 625]
{Michael D.~M.~Bader}

\institute 
{
  Practicum in Sociological Research (SOCY 625)
}
% - Use the \inst command only if there are several affiliations.
% - Keep it simple, no one is interested in your street address.

\date % (optional)
{Week 3}

\subject{Practicum in Sociological Research Slides}
% This is only inserted into the PDF information catalog. Can be left
% out.

% If you have a file called "university-logo-filename.xxx", where xxx
% is a graphic format that can be processed by latex or pdflatex,
% resp., then you can add a logo as follows:

%\logo{\includegraphics[height=1cm]{../../Images/au_logo_50by51px}}
%\logo{\includegraphics[height=1cm]{../../Images/au_logoname_300}}
\logo{\includegraphics[height=1cm]{/Users/bader/work/Presentations/Images/au_logoname_300}}

\subject{Intro to Survey Methods}
% This is only inserted into the PDF information catalog. Can be left
% out. 



% If you have a file called "university-logo-filename.xxx", where xxx
% is a graphic format that can be processed by latex or pdflatex,
% resp., then you can add a logo as follows:

% \pgfdeclareimage[height=0.5cm]{university-logo}{university-logo-filename}
% \logo{\pgfuseimage{university-logo}}



% Delete this, if you do not want the table of contents to pop up at
% the beginning of each subsection:
%\AtBeginSubsection[]
%{
%  \begin{frame}<beamer>{Outline}
%    \tableofcontents[currentsection,currentsubsection]
%  \end{frame}
%}


% If you wish to uncover everything in a step-wise fashion, uncomment
% the following command: 

%\beamerdefaultoverlayspecification{<+->}


\begin{document}

\begin{frame}
  \titlepage
\end{frame}

\section{Coverage Error}

\begin{concept}{undercoverage}{extent of target population \emph{missed} by sampling frame}\end{concept}

\begin{concept}{ineligible Units}{units included in the sampling frame that are not part of the target population}\end{concept}

\begin{frame}{Definition of Coverage Error}
\[
\overline{Y_c} - \overline{Y} = \frac{U}{N}\left(\overline{Y_c} - \overline{Y_u}\right)
\]

\begin{description}[noitemsep]
\item[$\overline{Y_c}$] Mean of the \emph{covered} population
\item[$\overline{Y}$] True underlying mean of the target population
\item[$U,N$] Number of uncovered units; total population of frame
\item[$\overline{Y_u}$] Mean of the \emph{uncovered} population
\end{description}
\end{frame}

\begin{frame}
\begin{center}
\includegraphics[scale=.8]{../images/noCoverageError.pdf}
\end{center}
\end{frame}

\begin{frame}
\begin{center}
\includegraphics[scale=.8]{../images/highCoverageError.pdf}
\end{center}
\end{frame}

\begin{frame}
\begin{center}
\includegraphics[scale=.8]{../images/highBiasedCoverage.pdf}
\end{center}
\end{frame}

\section{Probability Samples}

\begin{concept}{probability sample}{a sample taken using some form of randomization to select elements from the sample frame}\end{concept}

\begin{concept}{simple random sample}{every element in a frame has an equal probability of selection into the sample}\end{concept}

\begin{frame}
\begin{center}
\footnotesize
\begin{minipage}{.4\textwidth}
\begin{tabular}{lr}
\multicolumn{2}{l}{\textbf{Random start = 2}} \\
\multicolumn{2}{l}{\textbf{Sample prop.\ 1/5}} \\
\textbf{Room \#} & \textbf{Selected?} \\ \toprule
101 & \\
102 & Y \\
103 & \\
104 & \\ 
105 & \\
106 & \\ 
107 &Y  \\
108 & \\
201 & \\
202 & \\
203 & \\
204 &Y \\ 
205 & \\
206 & \\ 
207 & \\
208 &\\ \bottomrule
\end{tabular}
\end{minipage}\begin{minipage}{.4\textwidth}
\begin{tabular}{lr}
\multicolumn{2}{l}{} \\
\multicolumn{2}{l}{} \\
\textbf{Room \#} & \textbf{Selected?} \\ \toprule
701 & \\
702 &  \\
703 & \\
704 &Y  \\ 
705 & \\
706 & \\ 
707 & \\
708 &\\
801 & Y \\
802 &  \\
803 & \\
804 & \\ 
805 & \\
806 & Y \\ 
807 & \\
808 &\\ \bottomrule
\end{tabular}
\end{minipage}
\end{center}
\end{frame}

\begin{frame}{Area Probability Samples}
Sample places $\longrightarrow$ Sample housing units $\longrightarrow$ Sample people
\end{frame}

\begin{frame}
\begin{tabular}{lr}
\textbf{Building} & \textbf{Pop.} \\ \toprule
Dorm 1 & 64 \\
Dorm 2 & 64 \\
Dorm 3 & 64 \\
Dorm 4 & 200 \\
Dorm 5 & 200 \\
Dorm 6 & 400 \\
Dorm 8 & 400 \\
Dorm 9 & 400 \\
Dorm 10 & 8 \\ \bottomrule
\end{tabular}
\end{frame}


\begin{frame}
\begin{tabular}{lrr}
\textbf{Building} & \textbf{Pop.} & \textbf{Sampled?} \\ \toprule
Dorm 1 & 64 & \\
Dorm 2 & 64 & Y\\
Dorm 3 & 64 & Y \\ \midrule
Dorm 4 & 200& Y  \\
Dorm 5 & 200 & \\ \midrule
Dorm 6 & 400 & \\
Dorm 8 & 400 & Y \\
Dorm 9 & 400 & \\ \midrule
Dorm 10 & 8 & Y\\ \bottomrule
\end{tabular}
\end{frame}

\begin{frame}
\textbf{DC Area Survey}\\[2em]

Sample nhoods $\longrightarrow$ Sample addresses $\longrightarrow$ Sample people

\end{frame}
\end{document}


