\documentclass[11pt]{homework}


\title{DC Area Population and Survey Design}
\assignment{Homework Exercises, Week 3}
\duedate{February 11, 2018}
\coursetitle{Practicum in Sociological Research}
\coursenumber{SOCY 625}
\author{Prof. Michael Bader}



%\printanswers
\begin{document}
\maketitle 

\begin{questions}
\uplevel{{\bfseries Use the data from census tracts at \href{https://mdmb-mbweb.s3.amazonaws.com/media/courses/research-practicum/dcarea_census_tract_data.csv}{this link} (variable labels can be found \href{https://mdmb-mbweb.s3.amazonaws.com/media/courses/research-practicum/dcareaCensusVariableNames.pdf}{here}) to answer the following questions.} These data represent the population of the DCAS, which comprises residents in the District, Montgomery and Prince George's Counties in Maryland, and Arlington and Fairfax Counties in Virginia in addition to the independent cities of Alexandria, Falls Church, and Fairfax. Residents 18 and older from these counties represents the target population of the DCAS2018. }

\question[1] What is the size of the target population in the DC area? 

\question[2] Among census tracts, what is the average proportion of residents who identify as: 
\begin{parts}
\part non-Hispanic white alone; 
\part non-Hispanic black alone; 
\part Hispanic;
\part Asian alone or Pacific Islander alone;
\end{parts}

\question[1] What is the average percent poverty in tracts in the DC area? What is the standard deviation of percent poverty?

\question[4] Identify \emph{at least two} other characteristics of DC-area neighborhoods and describe them. 

\uplevel{\bfseries Think about your own research topic and respond to the following items.}

\question[1] Name the \emph{construct} you would like to measure

\question[3] For that construct, identify \emph{at least two} issues you might have measuring that construct. You should think about your answer in terms of measurement inference. 

\question[2] What sources of \emph{measurement} bias concern you? Answer for your construct specifically?

\question[2] What sources of \emph{representational} bias concern you? Answer for your construct specifically; think about who might be missed in our sample design and whether that might bias the questions that you ultimately ask? 

\question[1] Identify two Census items that you might use to test the construct validity of your measures. 
\begin{parts}
\part[2] Analyze the mean and standard deviation (or proportion, if categorical) of those characteristics in census tracts. 
\part[2] Explain what those characteristics would lead you to expect to find regarding the construct that you wish to measure. 
\end{parts}

\end{questions}
\end{document}
