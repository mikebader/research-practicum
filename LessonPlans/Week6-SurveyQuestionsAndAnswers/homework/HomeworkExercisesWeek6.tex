\documentclass[11pt]{homework}


\title{Question Testing}
\assignment{Homework Exercises, Week 6}
\duedate{March 4, 2018}
\coursetitle{Practicum in Sociological Research}
\coursenumber{SOCY 625}
\author{Prof. Michael Bader}



%\printanswers
\begin{document}
\maketitle 

\noindent Note: you may submit your response to Question~\ref{q:cognitiveiw} \emph{in class} on Monday, March 5. Please submit all other responses on Blackboard by Sunday, March 4 (as usual). 

\begin{questions}
\uplevel{\bfseries Answer the following questions on pages 287-290 of Groves, et al. (2009).}

\question[1.5] Question 1. 

\question[1] Question 3. 

\question[2] Question 4. 

\question[3] Question 6 (a-c). 

\question[2] Question 10 (a-b). For part (b), explain what the value of Chronbach's alpha would mean for this scale. 

\question[2] Question 11. 

\uplevel{\bfseries Answer the following based on a question or questions that \emph{you design} to put on the DCAS 2018. You \emph{may} use questions used in previous surveys, with appropriate attribution to the source.}

\question[1] Write the question(s) that you propose to include along with response categories (if appropriate). 

\question[3] Decide if your question is a sensitive behavioral question, a nonsensitive behavioral question, or an attitude question. After determining this, please go through the checklist that I handed out and assess each item on the checklist for each question. Write out your own assessment for each item in your response. 

\question Find at least two other students \emph{in our class} and conduct a cognitive interview of your question with them. Write a report (approx.\ 300-600 words) that 
\label{q:cognitiveiw}
\begin{parts}
\part[4] describes what you learned about your question from the cognitive interviews, and 
\part[4] what you plan to do to improve the question based on that data 
\end{parts}




\end{questions}
\end{document}
