\documentclass[]{beamer}
%\documentclass{article}
%\usepackage{beamerarticle}

\mode<presentation>
{
  \usetheme{Warsaw}
  % or ...

  \setbeamercovered{transparent}
  % or whatever (possibly just delete it)
}


\usepackage[english]{babel}
% or whatever

\usepackage[utf8]{inputenc}
% or whatever

\usepackage{times}
\usepackage[T1]{fontenc}
\usepackage{tikz}
\usetikzlibrary{arrows.meta}

%%%%%%%%%%%%%%%%%%%%%%%%%%%%%%%%%%
%%                         PACKAGES                      %%
%%%%%%%%%%%%%%%%%%%%%%%%%%%%%%%%%%
\usepackage{hyperref,ctable}
\usepackage{graphicx}
\usepackage{tikz}                    % For flowchart
\usetikzlibrary{shapes,arrows} % For flowchart


%%%%%%%%%%%%%%%%%%%%%%%%%%%%%%%%%%
%%                           COLORS                        %%
%%%%%%%%%%%%%%%%%%%%%%%%%%%%%%%%%%
%AU COLORS
\definecolor{aublue}{RGB}{25,33,129}
\definecolor{aured}{RGB}{244,28,31}

%Red
%\setbeamercolor{titlelike}{bg=aured,fg=white}
\setbeamercolor{structure}{bg=black!25!aured, fg=aured}
%\setbeamercolor*{palette primary}{fg=white,bg=aured}
%\setbeamercolor*{palette quaternary}{fg=white,bg=aublue}

%Blue
\setbeamercolor{titlelike}{bg=aublue,fg=white}
%\setbeamercolor{structure}{bg=black!25!aublue, fg=aublue}
\setbeamercolor*{palette primary}{fg=white,bg=aublue}
\setbeamercolor*{palette quaternary}{fg=white,bg=black!75!aublue}

\setbeamercolor{local structure}{fg=aured,bg=gray!60!aured}
\setbeamercolor{alerted text}{fg=aured}

\newenvironment{concept}[1]%
	{
	\setbeamercolor{background canvas}{bg=aured!10!white}%
	\setbeamercolor{frametitle}{bg=aured}%
	\setbeamercolor{frametitle right}{bg=aured}
	\setbeamercolor{alerted text}{fg=aured}%
	\begin{frame}{Concept}%
	\alert{\bfseries \large #1\\[2em]}}{%
	\end{frame}%
	}


%%%%%%%%%%%%%%%%%%%%%%%%%%%%%%%%%%
%%                         GRAPHICS                       %%
%%%%%%%%%%%%%%%%%%%%%%%%%%%%%%%%%%
%\graphicspath{{/Users/bader/work/Presentations/Images/}}}}


%%%%%%%%%%%%%%%%%%%%%%%%%%%%%%%%%%
%%                        COMMANDS                       %%
%%%%%%%%%%%%%%%%%%%%%%%%%%%%%%%%%%
\newcommand{\strong}[1]{\textbf{#1}}
\AtBeginSection[]{
  \begin{frame}
  \vfill
  \centering
  \begin{beamercolorbox}[sep=8pt,center,shadow=true,rounded=true]{title}
    \usebeamerfont{title}\insertsectionhead\par%
  \end{beamercolorbox}
  \vfill
  \end{frame}
}



%%%%%%%%%%%%%%%%%%%%%%%%%%%%%%%%%%
%%                     PRESENTATION                   %%
%%%%%%%%%%%%%%%%%%%%%%%%%%%%%%%%%%
\title{Error}

\author[Bader--SOCY 625]
{Michael D.~M.~Bader}

\institute 
{
  Practicum in Sociological Research (SOCY 625)
}
% - Use the \inst command only if there are several affiliations.
% - Keep it simple, no one is interested in your street address.

\date % (optional)
{Week 2}

\subject{Practicum in Sociological Research Slides}
% This is only inserted into the PDF information catalog. Can be left
% out.

% If you have a file called "university-logo-filename.xxx", where xxx
% is a graphic format that can be processed by latex or pdflatex,
% resp., then you can add a logo as follows:

%\logo{\includegraphics[height=1cm]{../../Images/au_logo_50by51px}}
%\logo{\includegraphics[height=1cm]{../../Images/au_logoname_300}}
\logo{\includegraphics[height=1cm]{/Users/bader/work/Presentations/Images/au_logoname_300}}

\subject{Intro to Survey Methods}
% This is only inserted into the PDF information catalog. Can be left
% out. 



% If you have a file called "university-logo-filename.xxx", where xxx
% is a graphic format that can be processed by latex or pdflatex,
% resp., then you can add a logo as follows:

% \pgfdeclareimage[height=0.5cm]{university-logo}{university-logo-filename}
% \logo{\pgfuseimage{university-logo}}



% Delete this, if you do not want the table of contents to pop up at
% the beginning of each subsection:
%\AtBeginSubsection[]
%{
%  \begin{frame}<beamer>{Outline}
%    \tableofcontents[currentsection,currentsubsection]
%  \end{frame}
%}


% If you wish to uncover everything in a step-wise fashion, uncomment
% the following command: 

%\beamerdefaultoverlayspecification{<+->}


\begin{document}

\begin{frame}
  \titlepage
\end{frame}

\begin{frame}
\begin{figure}[h!]
\begin{center}
\includegraphics[height=7cm]{../images/GrovesCh2Fig2.pdf}
\end{center}
\end{figure}
\end{frame}

\section{Error in Measurement}

\begin{frame}
\begin{figure}[h!]
\begin{center}
\includegraphics[height=7cm]{../images/GrovesCh2Fig2Measurement.pdf}
\end{center}
\end{figure}
\end{frame}

\begin{frame}{Components of Measurement}
\begin{description}
\item[Construct] True concept of interest \pause
\item[Measurement] How we attempt to assess the construct and turn it into data (e.g.\ survey questions) \pause
\item[Response] How people that we survey actually answer the items (questions) that we use to measure the construct \pause
\item[Edited response] How we modify responses to put them into a form so that we can analyze the constructs across a population \pause
\begin{description}
\item[Field editing] Checking or coding responses while the survey is being conducted
\item[Coding] Putting collected data into units or values that can be measured (almost always done after field period is complete)
\end{description}
\end{description}
\end{frame}

\begin{frame}
\begin{figure}[h!]
\begin{center}
\includegraphics[height=7cm]{../images/GrovesCh2Fig2Measurement.pdf}
\end{center}
\end{figure}
\end{frame}

\begin{frame}{Measurement Errors}
\begin{description}
\item[Validity errors] \textit{(between Construct \& Measurement)} discussed above
\item[Measurement error] \textit{(between Measurement \& Response)} Construct \emph{can} be validly measured with the items (questions), but respondents do not answer the question as the question was intended to answer (e.g., missing the question $2+2$ does not make the question invalid, but measurement error would be the result of an arithmetic mistake)
\item[Processing error] \textit{(between Response \& Edited response)} Coding or data entry errors, or the codes that we create don't represent what the respondents intended
\end{description}
\end{frame}

\section{Error in Representation}

\begin{frame}
\begin{figure}[h!]
\begin{center}
\includegraphics[height=7cm]{../images/GrovesCh2Fig2Representation.pdf}
\end{center}
\end{figure}
\end{frame}

\begin{frame}{Components of Representation}
\begin{description}
\item[Target population] The real body of people whom we want to represent \pause
\item[Sampling frame] Enumeration of target population from which we draw our sample of respondents \pause
\item[Sample] Units \emph{selected} to be part of the survey \pause
\item[Respondents] Units\slash people who \emph{respond} to the survey \pause
\item[Post-survey adjustments] Weighting used to make \emph{respondents} represent the \emph{target population} 
\end{description}
\end{frame}

\begin{frame}
\begin{figure}[h!]
\begin{center}
\includegraphics[height=7cm]{../images/GrovesCh2Fig2Representation.pdf}
\end{center}
\end{figure}
\end{frame}

\begin{frame}{Errors of Representation}
\begin{description}
\item[Coverage error] \textit{(between Target Population \& Sampling Frame)} Error introduced by not including people in the sampling frame, but who should be in the sampling frame (i.e., eligible to be included in the sample but not)
\item[Sampling error] \textit{(between Sampling Frame \& Sample)} we will come back to this
\item[Response error] \textit{(between Sample \& Respondents)} error introduced by people who did not respond to interview requests even though they were sampled
\item[Adjustment error] \textit{(between Respondents \& Post-survey adjustments)} Errors introduced by not properly weighting the responses received back to the target population
\end{description}
\end{frame}





\end{document}


