\documentclass[11pt]{homework}

\usepackage{enumitem}

\title{Writing Good Questions}
\assignment{Week 6}
\duedate[]{}
\coursetitle{Practicum in Sociological Research}
\coursenumber{SOCY 625}
\author{Prof. Michael Bader}



%\printanswers
\begin{document}
\maketitle 

\section{Nonsensitive Questions about Behavior}
On page 243, Groves, et al. write out a list of guidelines (a checklist of sorts) for evaluating behavior-related questions on non-sensitive topics

\begin{enumerate}[noitemsep]
\item Create comprehensive and mutually exclusive response categories
\item Make question as specific as possible
\item Use familiar words
\item Lengthen questions to add memory cues to improve recall (suggest to respondent what you want them to consider)
\item When forgetting is likely, use aided recall (e.g. use subcategories to questions)\hrule\vspace{.3em}
\item Use a diary
\item Use a life history calendar
\item Ask respondent to use household records to confirm answers
\item Allow a proxy to report for the respondent
\end{enumerate}

\section{Sensitive Questions about Behavior}
On page 246, Groves, et al.\ write out a list of guidelines for questions about sensitive behaviors

\begin{enumerate}[noitemsep]
\item Use open rather than closed questions for eliciting the frequency of sensitive behaviors
\item Use long rather than short questions
\item Use familiar words
\item Deliberately load the question to reduce misreporting (write something that shows that certain behaviors are ``okay'' to reduce motivated misreporting)
\item Ask about long periods before asking about more recent behaviors
\item Embed sensitive question among other sensitive items to make it stand out less
\item Add items to assess the sensitivity of behavioral items
\hrule\vspace{.3em}
\item Allow the respondent to administer the survey to him-\slash herself 
\item Collect data in a diary
\item Collect validation data 
\end{enumerate}

\clearpage 

\section{Attitude Questions}
On page 248, Groves, et al.\ write out a list of guidelines for attitude questions
\begin{enumerate}[noitemsep]
\item Specify the attitude object clearly
\item Measure the strength of the attitude, if necessary using separate items (to avoid double-barreled questions)
\item Use bipolar items except when they might miss key information
\item Carefully consider the alternatives mentioned in the question (they have a big impact on the answers)
\item In measuring change over time or over different groups, ask the same questions
\item When asking both general and specific questions about a topic, ask general items first
\item When asking about multiple items, start with the least popular one
\item Use closed questions for measuring attitudes
\item Use five- or seven-point scales and label every point
\item Start with the end of the scale that is least popular
\item Use analog devices (such as thermometers) to collect more detailed scale information
\item Use ranking only if the respondents can see all alternatives; otherwise use pair-wise comparisons
\item Get ratings for every item of interest; do not use check-all-that-apply items
\end{enumerate}

\end{document}
