\documentclass[11pt]{lecturenotes}

\newcommand{\code}[1]{\texttt{#1}}


\title{Inference and Error in Survey Design}
\author{Michael Bader}
\week{2}
\lesson{1}
\coursenumber{SOCY 625}
\coursetitle{Practicum in Sociological Research}


\begin{document}
\maketitle

\begin{objectives}{
\item Describe the process of collecting data with surveys
\item Define the two types of inference in survey research
\item Identify and explain sources of error in surveys 
\item Explain the concept of survey error}{
\item explain the principles of survey design including question design, sampling, statistical inference, and sources of error;
\item develop and test questions to answer a specific research hypothesis; 
}
\end{objectives}

\section[10]{Review}
Any questions from last week? 

\section{Inference}
\slide
\concept{Inference}{``Formal logic that permits description of unobserved phenomena based on observed phenomena.'' (Groves, et al., 40}

\subsection[7.5]{Types of Inference}
\marginpar{\vspace{1em} \itshape \raggedleft Slide}
\begin{description}
\item[Inference by Representation:] One inference comes from the fact that we only measure a small fraction of the population that we care about. 
\begin{itemize}
\item We cannot \emph{observe} the entire population (in most cases), but we can \emph{infer} aspects of the population based on a well-designed and executed sample. 
\item Therefore, we need to \emph{infer} from the sample of respondents what the truth is across an entire population
\item To do this, we rely on probability theory and statistics, but we will come back to that. 
\end{itemize}
\item[Inference by Measurement:] Another inference comes from trying to measure the ``unobserved phenomena'' inside people's heads. 
\end{description}

\section{Measurement Inference}
\subsection[10]{Defining Measurement Inference}
Let's think about the things that we want to study and make claims about: attitudes, beliefs, perceptions and opinions

\slide
\emph{Examples:} How traumatized are black students about racist events? How open to ``alternative'' sexual experiences are people? How much patriotism or nationalism do people feel? 

\slide
\concept{construct}{elements of information that are sought by the researcher (Groves, et al., 41)}

\slide
Go back to examples from above; what are the constructs? (trauma, openness, patriotism\slash nationalism)

How often might you have had a difficult time expressing yourself and your own feelings, attitudes, or beliefs about a subject? Are they real? Are they always easy to communicate? Now try to imagine getting those real but unobservable thoughts from an entire population. 

\slide
\begin{center}
\begin{tikzpicture}[scale=1]
\node [left] at (0,0) {Construct};
\draw [-{Latex[length=2mm]}, thick] (0,0) -- (3,0);
\node [right] at (3,0) {Question};
\node [align=left, right]  at (4,-1) {\small Something gets lost\\in translation};
\path [-Latex] (4,-1) edge [bend left=30] (1.5,0) {};
\draw [red,dashed] (1.5,0) ellipse (1.5 and .5);
\node [align=center,below,red] at (1.5,-.5) {Inference};
\end{tikzpicture}
\end{center}

Since we can't extract thoughts, emotions, attitudes, or opinions from people's heads, we are forced to \emph{infer} what those thoughts, emotions, attitudes, or opinions are. 

\subsection[20]{Validity \& Reliability}
\slide
\concept{validity}{The accuracy with which our measures reflect the true underlying value}

\slide
Therefore, \emph{construct validity} refers to the degree that our measures (questions on a survey) measure the true underlying information sought by the researcher. 

Going back to our picture: the less that gets lost in translation, the \emph{better our inference} and the \emph{higher the construct validity}

\slide
\begin{center}
\begin{tikzpicture}[scale=1]
\node [left] at (0,0) {Construct};
\draw [-{Latex[length=2mm]}, thick] (0,0) -- (3,0);
\node [right] at (3,0) {Question};
\node [align=left, right]  at (4,-1) {\small Validity: \emph{how much} gets lost\\in translation};
\path [-Latex] (4,-1) edge [bend left=30] (1.5,0) {};
\draw [red,dashed] (1.5,0) ellipse (1.5 and .5);
\node [align=center,below,red] at (1.5,-.5) {Inference};
\end{tikzpicture}
\end{center}

A problem: we can't \emph{measure} the construct, hence the inference. If we can't measure the construct, how do we know if our measure is valid? We could examine wither the measures that we collect ``move with'' (are correlated with) other variables in the predicted direction, if we ask people over and over again do they give the same result, does our rank-order confirm other types of data collection?

This is where \emph{psychometrics} comes in: the branch of social science that studies measurement

\slide
\[
\left.
\begin{array}{l}
\text{I ask you one question} \\
\text{I ask you five questions}
\end{array}\right\} \text{Which measures apptitude better?}
\]

Each \emph{item} has error. Ideally, it will be small; but averaging across items measuring the same construct will reduce the error (since the mean minimizes error)

A related idea is how stable the measures are from different trials within the same person.

\slide
\concept{reliability}{How closely repeated measurements of the same construct yield the same result}

If you take a test one day and take the exact same test the next day (without studying in between), will you get different scores? Probably a little bit 

Has your ability changed between the two days? Probably not

The larger the difference between tests (trials) within individuals across a population of people, the lower the reliability of the measure. 

\slide
\begin{center}
\begin{tabular}{rp{3in}}
\textbf{reliability} & degree to which measurement of the same construct yields the same answer over repeated trials \emph{with the same person}\\[.5em]
\textbf{validity} & degree to which the measure accurately represents the underlying construct
\end{tabular}
\end{center}

The importance of the reliability and the validity of measures depends, to a large degree, on the purpose of the study

\slide
\textbf{\textsc{It is really, really important that measurement error should not be biased!!!}}

\section[5]{Representational Inference}
The other set of inference that we must use to characterize \emph{a population} is how we represent that population with \emph{a sample}.

\slide
\begin{center}
\begin{tikzpicture}[scale=1]
\node [left] at (0,0) {Population};
\draw [-{Latex[length=2mm]}, thick] (0,0) -- (3,0);
\node [right] at (3,0) {Sample};
\node [align=left, right]  at (4,-1) {\small Some people are\\ not measured};
\path [-Latex] (4,-1) edge [bend left=30] (1.5,0) {};
\draw [red,dashed] (1.5,0) ellipse (1.5 and .5);
\node [align=center,below,red] at (1.5,-.5) {Inference};
\end{tikzpicture}
\end{center}

We do have a set of tools, probability and statistics, designed to handle this set of inference so that we can take a sample of people and make claims about the larger population from which they were drawn. 

We want to make sure that the sample \emph{represents} as similar of a cross-section of the population, within known parameters, as we possibly can. 

Validity in this inference will come from our ability to design and conduct a proper sample, and to interview the people we do sample. 

\section[10]{Summary}
\slide
\begin{itemize}
\item Inference: measuring unobservable phenomena with observed items
\item Measurement inference involves trying to measure the unobserved phenomena as closely as possible
\begin{itemize}
\item The more closely we measure the true underlying phenomenon, the more valid our measures
\item The more we can trust a measure taken at one time represents the possible measures given under the same conditions, the more reliable our measures
\end{itemize}
\item Representational Inference involves trying to characterize a population with a sample
\end{itemize}

Here is the picture that Groves, et al. used to illustrate the two paths of inference. 

\clearpage
On the left we have the measurement. The process of asking \& answering questions. Inference goes from response to construct (we \emph{infer} the construct based on responses). 

On the right we have representation. The process of deciding who to ask about the constructs. Inference goes from target (we \emph{infer} the population level of some construct based on our adjusted set of respondents)

\begin{figure}[h!]
\begin{center}
\includegraphics{../images/GrovesCh2Fig2.pdf}
\end{center}
\end{figure}
%
%
%
%\section{Homework}
%\begin{itemize}
%\item Groves, et al.\ pp.\ 36 \& 37: answer \textbf{one of} Questions 1-6 and Questions 7 \& 9 (three questions total)
%\item Groves, et al.\ pp.\ 65-67: answer  Questions 1(a-c), 2, 4, 5 (pick four from a-i), \& 6 (pick four from a-k)
%\item Download R and RStudio
%\item Load DCAS2016 data from the web and from a local file (submit R scripts)
%\end{itemize}



\end{document}
