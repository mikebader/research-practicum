\documentclass[11pt]{lecturenotes}

\newcommand{\code}[1]{\texttt{#1}}

\title{Error in Survey Design}
\author{Michael Bader}
\week{2}
\lesson{2}
\coursenumber{SOCY 625}
\coursetitle{Practicum in Sociological Research}


\begin{document}
\maketitle

\begin{objectives}{
\item Identify and explain sources of error in surveys 
\item Explain the concept of survey error}{
\item explain the principles of survey design including question design, sampling, statistical inference, and sources of error;
\item develop and test questions to answer a specific research hypothesis; 
}
\end{objectives}

\section[5]{Errors in Survey Design}
\slide
Groves, et al.\ conceptualized the two paths of inference in Figure 2.2. 

On the left we have the measurement. The process of asking \& answering questions. Inference goes from response to construct (we \emph{infer} the construct based on responses). 

On the right we have representation. The process of deciding who to ask about the constructs. Inference goes from target (we \emph{infer} the population level of some construct based on our adjusted set of respondents)

\begin{figure}[h!]
\begin{center}
\includegraphics{../images/GrovesCh2Fig2.pdf}
\end{center}
\end{figure}

On each side, they discuss where errors creep into the process and so we will go through those now. 

\clearpage

\section{Error in Measurement}
\slide
Let's go through the measurement side (left side) of the figure and define each of these terms and see where error comes into play. 

\begin{figure}[h!]
\begin{center}
\includegraphics[scale=.7]{../images/GrovesCh2Fig2Measurement.pdf}
\end{center}
\end{figure}

\slide
\vspace{-2em}
\subsection[5]{Components of Measurement}
\begin{description}
\item[Construct] True concept of interest
\item[Measurement] How we attempt to assess the construct and turn it into data (e.g.\ survey questions)
\item[Response] How people that we survey actually answer the items (questions) that we use to measure the construct
\item[Edited response] How we modify responses to put them into a form so that we can analyze the constructs across a population
\begin{description}
\item[Field editing] Checking or coding responses while the survey is being conducted
\item[Coding] Putting collected data into units or values that can be measured (almost always done after field period is complete)
\end{description}
\end{description}
\vspace{-1em}
Careful design considerations go into each of each of these steps. 

\subsection[10]{Errors in Measurement}
\slide
Errors exist every time we move from one step to the next

\begin{description}
\item[Validity errors] \textit{(between Construct \& Measurement)} discussed above
\item[Measurement error] \textit{(between Measurement \& Response)} Construct \emph{can} be validly measured with the items (questions), but respondents do not answer the question as the question was intended to answer (e.g., missing the question $2+2$ does not make the question invalid, but measurement error would be the result of an arithmetic mistake)
\item[Processing error] \textit{(between Response \& Edited response)} Coding or data entry errors, or the codes that we create don't represent what the respondents intended
\end{description}
\textbf{validity errors} -- those that we just discussed

\clearpage

\section{Error in Representation}
\slide
Now let's turn to the representation (right) side of the figure. To repeat what I said before: 

On the right we have representation. The process of deciding who to ask about the constructs. Inference goes from target (we \emph{infer} the population level of some construct based on our adjusted set of respondents)


\begin{figure}[h!]
\begin{center}
\includegraphics[scale=.7]{../images/GrovesCh2Fig2Representation.pdf}
\end{center}
\end{figure}

\subsection[5]{Components of Representation}
\slide
\begin{description}
\item[Target population] The real body of people whom we want to represent
\item[Sampling frame] Enumeration of target population from which we draw our sample of respondents
\item[Sample] Units \emph{selected} to be part of the survey
\item[Respondents] Units\slash people who \emph{respond} to the survey
\item[Post-survey adjustments] Weighting used to make \emph{respondents} represent the \emph{target population}
\end{description}

\subsection[10]{Errors of Representation}
\slide
\begin{description}
\item[Coverage error] \textit{(between Target Population \& Sampling Frame)} Error introduced by not including people in the sampling frame, but who should be in the sampling frame (i.e., eligible to be included in the sample but not)
\item[Sampling error] \textit{(between Sampling Frame \& Sample)} we will come back to this
\item[Response error] \textit{(between Sample \& Respondents)} error introduced by people who did not respond to interview requests even though they were sampled
\item[Adjustment error] \textit{(between Respondents \& Post-survey adjustments)} Errors introduced by not properly weighting the responses received back to the target population
\end{description}

\clearpage
\section[25]{Sampling Error}
\clearpage

\section[20]{Descriptive Statistics in R}

See sheet with R notebook containing exercise.










\section[5]{Homework}
Please be sure to use Piazza for discussions. I \emph{will not} answer questions this week unless classmates attempt to answer the questions first. I will post the slides and a handout going through the R exercise we did on Piazza. 

For this week: 
\begin{itemize}
\item Groves, et al.\ pp.\ 95 \& 96: answer questions 2, 4, 5, 6
\item Describe how Roberts, et al.\ (2004) created their sample and what they did to minimize representation errors
\item Explain the disagreement between the Iraqi Body Count and the Roberts, et al.\ team \emph{based on sampling theory}
\item Fill out the following descriptive table for DCAS2016 respondents and include the R script to show how you came to the numbers that you did
\end{itemize}



\end{document}
