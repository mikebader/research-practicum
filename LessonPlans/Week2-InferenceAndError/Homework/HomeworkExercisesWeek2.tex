\documentclass[11pt]{homework}


\title{Introduction to Survey Methods and to R}
\assignment{Homework Exercises, Week 2}
\duedate{February 5, 2018}
\coursetitle{Practicum in Sociological Research}
\coursenumber{SOCY 625}
\author{Prof. Michael Bader}



%\printanswers
\begin{document}
\maketitle 

\begin{questions}
\uplevel{From Chapter 3 in Groves, et al.\ pp.\ 95 \& 96, answer:}

\question Question 2

\question Question 4

\question Question 5

\question Question 6

\uplevel{Using the Roberts, et al.\ (2004) and Iraqi Body Count readings:}

\question  Describe how Roberts, et al.\ (2004) created their sample and what they did to minimize representation errors

\question Explain the disagreement between the Iraqi Body Count and the Roberts, et al.\ team \emph{based on sampling theory}

\uplevel{Using the DCAS2016 data that you downloaded last week}

\question Fill out the following descriptive table for DCAS2016 respondents and include the R script to show how you came to the numbers that you did\\[1em]

\begin{tabular}{lrr}
\textbf{Variable} & \textbf{Mean} & \textbf{Standard Deviation}\\ \toprule
Age & & \\
Gender & & \\
\quad Male & & \\
\quad Female & & \\
Race & & \\
\quad Prop.\ White & & \\
\quad Prop.\ Black & & \\
\quad Prop.\ Hispanic & & \\
\quad Prop.\ Asian \\
Number of years lived in neighborhood & & \\
Proportion affected by police \emph{a lot} & & \\\bottomrule


\end{tabular}


\end{questions}
\end{document}
